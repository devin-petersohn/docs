% This is LLNCS.DEM the demonstration file of
% the LaTeX macro package from Springer-Verlag
% for Lecture Notes in Computer Science,
% version 2.4 for LaTeX2e as of 16. April 2010
%
\documentclass{llncs}
%
\usepackage{makeidx}  % allows for indexgeneration
\usepackage{amsmath}
\usepackage{listings}
\usepackage{algorithm}
\usepackage{algorithmic}

\lstdefinelanguage{scala}{morekeywords={class,object,trait,extends,with,new,if,while,for,def,val,var,this},
otherkeywords={->,=>},
sensitive=true,
morecomment=[l]{//},
morecomment=[s]{/*}{*/},
morestring=[b]"}
% Default settings for code listings
\lstset{frame=tb,language=scala,aboveskip=3mm,belowskip=3mm,showstringspaces=false,columns=flexible,basicstyle={\small\ttfamily}}
%
\begin{document}
%
\frontmatter          % for the preliminaries
%
\pagestyle{headings}  % switches on printing of running heads
\addtocmark{Distributed, In-Memory $k$-mer Counting}
\mainmatter              % start of the contributions
%
\title{Distributed, In-Memory $k$-mer Counting}
%
\titlerunning{Distributed, In-Memory $k$-mer Counting}  % abbreviated title (for running head)
%                                     also used for the TOC unless
%                                     \toctitle is used
%
\author{Frank~Austin~Nothaft, Anthony~D.~Joseph, \and David~A.~Patterson}
%
\authorrunning{Nothaft et al.} % abbreviated author list (for running head)
%
\institute{Department of Electrical Engineering and Computer Sciences\\
University of California, Berkeley \\ 
Berkeley, CA 94720, USA \\
\email{\{fnothaft, adj, pattrsn\}@berkeley.edu}}

\maketitle              % typeset the title of the contribution

\begin{abstract}
Calculating the relative abundances of $k$-mers is a key task in many alignment-free genomics
pipelines. However, $k$-mer counting is slow for 

Read error correction is a key step in many genome assembly protocols, but is computationally
expensive. Recent papers have proposed several strategies for approximating steps in the error
correction process. In this paper, we derive a rigorous generative model for identifying and correcting
errors in short reads. We implement our algorithm using the Apache Spark distributed computing
platform. Our implementation demonstrates a performant and exact approach to read error correction
that can scale efficiently to mammalian genomes.

\keywords{$k$-mer, distributed computing, map-reduce}
\end{abstract}

\section{Introduction}
\label{sec:introduction}

Read error correction is a key step in most genome assembly protocols, but is computationally
expensive~\cite{kelley10}. Recent work has explored mechanisms for performing error correction with
probabilistic data structures which consume less memory~\cite{song14,shi10,liu13,heo14}. However,
we assert that these techniques are unnecessary with the rise of distributed, in-memory computing
frameworks that scale easily on commodity hardware~\cite{zaharia10}.

In this paper, we present a distributed algorithm for performing read error correction, which is built using
the ADAM libraries for distributed genomics~\cite{massie13}. Our algorithm trains a model for read
errors using the abundance spectrum of quality score weighted $k$-mers (known as
$q$-mers~\cite{kelley10}). Given these probabilities, we estimate transition probabilities across classes
which contain correlated errors. Finally, per read, we perform error correction via a coordinate ascent
process. As our algorithms are implemented on top of the Apache Spark in-memory MapReduce
system~\cite{zaharia10}, we are able to achieve both scale up and scale out performance. Our
software is open source under the Apache 2 license.

We make the following contributions:

\begin{enumerate}
\item We implement a MapReduce based, in-memory $k$/$q$-mer counting engine.
\item We demonstrate a probabilistically rigorous, generative model for read error correction.
\item Our model is implemented in a scalable fashion on an efficient, MapReduce engine.
\end{enumerate}

\section{Background and Related Work}
\label{sec:background}

\section{Implementation}
\label{sec:implementation}

\subsection{Distributed $k$-mer Counting}
\label{sec:distributed-kmer-counting}

On top of a MapReduce platform, $k$-mer counting reduces down to the classical ``word count'' problem.
Our $k$-mer counting implementation uses the following simple code:

\begin{lstlisting}
def adamCountKmers(kmerLength: Int): RDD[(String, Long)] = {
  rdd.flatMap(r => {
    // cut each read into k-mers, and attach a count of 1L                                                                                                                                                  
    r.getSequence
	.toString
	.sliding(kmerLength)
	.map(k => (k, 1L))
  }).reduceByKey((k1: Long, k2: Long) => k1 + k2)
}
\end{lstlisting}

This MapReduce job is implemented on top of the Apache Spark platform, which provides the resilient
distributed dataset~(RDD) abstraction~\cite{zaharia12}. This abstraction presents a view of an array which
is distributed as partitions across machines. In the code above, we first perform a \texttt{flatMap}
transformation which cuts reads into $k$-mers of multiplicity 1. These intermediate $k$-mers are then
persisted in memory as an intermediate result. We then perform a \texttt{reduceByKey}, where the
$k$-mer string is used as the key. This stage opportunistically performs the reduction on local keys, before 
globally ``shuffling'' data between machines.

As noted in the rest of the paper, we use ``$q$-mers'', which were introduced by Kelley et
al~\cite{kelley10}. $q$-mers represent quality score weighted $k$-mers. Thus, while the multiplicity of a
$k$-mer seen in a read is 1, the multiplicity of a $q$-mer is bounded between $[0,1]$. The implementation
of our $q$-mer counter is a slight modification of our $k$-mer count implementation. Conceptually, the
$q$-mer counting process provides expectations for the \emph{true} counts of all $k$-mers, which makes
it highly useful for detecting erroneous $k$-mers.

\subsection{Erroneous $k$-mer Detection}
\label{sec:detection}

Given the $q$-mer spectrum, we apply an unsupervised clustering model to learn the
classifications of $q$-mers. For a sample with ploidy of $m$, we model $q$-mers as being drawn from
a selection of $m + 1$ Gamma distributions. $m - 1$ of these distributions model ``true'' $q$-mers
which are drawn from heterozygous sites, one distribution models ``true'' $q$-mers drawn from
homozygous sites, and one distribution models erroneous $q$-mers. While prior work by Kelley et
al.~\cite{kelley10} has used the Normal distribution to model the counts of ``true'' $q$-mers, it is
preferable to use the Gamma distribution, whose support is limited to $[0, \infty)$.

We describe the process we use for counting $q$-mers in~\S\ref{sec:distributed-kmer-counting}. Once
we have counted $q$-mers, we then fit our mixture model using the expectation-\linebreak
maximization~(EM) algorithm introduced by Almhana et al.~\cite{almhana06} We considered the
optimized algorithm for fitting mixtures of Gammas that was proposed by Schwander and
Nielsen~\cite{schwander13}, but chose the Almhana EM algorithm because the Schwander $k$-MLE
algorithm is slower for mixtures that contain fewer than eight components. We describe the
implementation of this algorithm in~\S\ref{sec:distributed-em}.

Once the mixture model has been fit, each $k$-mer can be assigned an error probability. This
probability is given in equation~\eqref{eq:p-err}, and is defined by the softmax likelihood that a $k$-mer
is drawn from the error distribution, given the $q$ weight of the $k$-mer.

\begin{equation}
\label{eq:p-err}
P(k \text{ is erroneous}) = \frac{\mathcal{L}(s = 0 | k)}{\sum_{i = 0}^{m + 1} \mathcal{L}(s = i | k)}
\end{equation}

Once the error probabilities have been calculated for all $k$-mers, we then filter the set of $k$-mers
to obtain the set of trusted $k$-mers. We apply the following filters, and place all $k$-mers that satisfy
these filters into a trie:

\begin{enumerate}
\item The $k$-mer is kept if it's error probability is below a user provided error threshold, and
\item The $k$-mer does not contain any IUPAC disambiguation codes.
\end{enumerate}

\subsection{Distributed EM Algorithms}
\label{sec:distributed-em}

The Spark MapReduce platform was designed to enhance the performance of iterative MapReduce
algorithms, which are commonplace in machine learning and related disciplines~\cite{zaharia10}. Spark
achieves this by providing the ability to hold datasets in memory between iterations. This improves
performance when compared to Hadoop, where intermediate results must be spilled to disk after every
map phase. Through the effective use of caching, we are able to implement high performance,
distributed EM on top of Spark, without the use of any approximations.

As mentioned in~\S\ref{sec:detection}, we use the EM algorithm for mixtures of Gamma distributions
which was introduced by Almhana et al~\cite{almhana06}. To fit a mixture of $i \in [0, \dots, n - 1]$
Gamma distributions with shape $\alpha_i$ and rate~(inverse of scale) $\beta_i$ to $n$ points, we
perform the following M updates during step $t$:

\begin{equation}
\label{eq:gamma-em-pi}
\pi_i^{(t + 1)} = \frac{1}{n} \sum_{j = 1}^n p(i | x_j, \Theta^{(t)})
\end{equation}
\begin{equation}
\label{eq:gamma-em-beta}
\beta_i^{(t + 1)} = \frac{\alpha_i^{(t)} \sum_{j = 1}^n p(i | x_j, \Theta^{(t)})}{\sum_{j = 1}^n x_j p(i | x_j,
\Theta^{(t)})}
\end{equation}
\begin{equation}
\label{eq:gamma-em-alpha}
\alpha_i^{(t + 1)} = \alpha_k^{(i - 1)} + \frac{1}{t} G_{\alpha_i}(X, \Theta^{(t)})
\end{equation}

The $G_{\alpha_i}(X, \Theta^{(t)})$ expression is defined as follows:

\begin{equation}
\label{eq:gamma-em-g}
G_{\alpha_i}(X, \Theta^{(t)}) = \frac{1}{n} \sum_{j = 1}^n [\log(x_j) - \log(\beta_i^{(t)}) -
\psi(\alpha_i^{(t)})] p(i | x_j, \Theta^{(t)})
\end{equation}

We initialize our distributions before running EM by running $k$-means, using the MLLib/MLI
implementation of $k$-means~\cite{sparks13}. We associate each point to the cluster centroid it is
closest to, and compute the mean~($\mu$) and standard deviation~($\sigma$) of the points associated with each
centroid. Given the $\mu$ and $\sigma$ per centroid, we initialize a Gamma distribution by solving
$\alpha = \frac{\mu}{\sigma}$ and $\beta = \frac{\alpha}{\mu}$. In pseudocode, our distributed
implementation of EM is as follows:

\begin{algorithm}
\caption{Run Distributed EM}
\label{alg:distributed-em}
\begin{algorithmic}
\STATE $data \leftarrow$ input dataset
\STATE $\Theta \leftarrow$ initial distribution parameters in mixture
\STATE $\pi \leftarrow$ initial distribution weights
\REQUIRE $data$ is cached
\STATE $iteration \leftarrow 1$
\REPEAT
\STATE $pointProbabilities \leftarrow data$.map($x_j \Rightarrow p(i | x_j, \Theta, \pi) \forall i$)
\STATE $pointProbabilities$.cache()
\STATE $nonNormalizedWeights \leftarrow pointProbabilities$.aggregate($(a, b) \Rightarrow a + b$)
\STATE $weightsAndPoints \leftarrow pointProbabilities$.zip($data$)
\STATE $pointProbabilities$.uncache()
\STATE $weightsAndPoints$.cache()
\STATE $betaDenominator \leftarrow weightsAndPoints$.map($(\tau_j, x_j) \Rightarrow x_j \tau_j$)\\
\hspace{\algorithmicindent}.aggregate($(a, b) \Rightarrow a + b$)
\STATE $G \leftarrow weightsAndPoints$.map($(\tau_j, x_j) \Rightarrow \tau_j^T [\log(x_j) - \log(\beta_i) - \psi(\alpha_i), \forall i]$)\\
\hspace{\algorithmicindent}.aggregate($(a, b) \Rightarrow a + b$)
\STATE $weightsAndPoints$.uncache()
\STATE $\pi \leftarrow$ softmax($nonNormalizedWeights$)
\FORALL{$(\alpha, \beta) \in \Theta, g \in G, b \in betaDenominator$}
\STATE $\beta \leftarrow \frac{\alpha nonNormalizedWeight}{b}$
\STATE $\alpha \leftarrow \alpha + \frac{g}{iteration}$
\ENDFOR
\STATE $iteration \leftarrow iteration + 1$
\UNTIL{converged}
\RETURN $\Theta$
\end{algorithmic}
\end{algorithm}

During this algorithm, we calculate the expected complete log \linebreak likelihood~(ECLL) of the
mixture. We run this algorithm until either a maximum iteration limit has been reached, or until we are
no longer making ``significant'' improvements in the ECLL. We provide the user with the ability to set
the early termination cut-off threshold for ECLL improvement.

\subsection{Base Transition Probability}
\label{sec:transition-probability}

Due to well-known biases in the sequencing process, the four nucleotides have different probabilities
of being sequenced incorrectly. Errors are known to be correlated within specific nucleotides, the
position of the base in the read, and empirical quality score values. These three \emph{covariates} are
used in the GATK's base quality score recalibration~(BQSR) process which measures the ``true'' error
probability that corresponds to a base quality score~\cite{depristo11}.

While base quality scores are a useful prior for the GATK's main goal~(specifically, variant calling),
their binary error/success nature is not as useful of a prior for read error correction. Instead, since we
are trying to predict whether a true base $b$ was incorrectly sequenced as $b'$, where $b, b' \in
\{ \text{A}, \text{C}, \text{G}, \text{T} \}$\footnote{In practice, the observed base, $b'$, may be
represented by any of the IUPAC base disambiguation codes. While we handle these cases in our
implementation, we do not include them here for notational simplicity. The corrected base, $b$, is
constrained to not be a disambiguation code.}, we desire our prior to be the transition probabilities from
$b' \rightarrow b$. These transition probabilities can be represented by categorical random variables
with four categories. We want to estimate these distributions per error covariate, where the error
covariate is defined by the sequenced base, the empirical base quality score, and the position of the
base in the read.

The transition probability for error covariate $C_{b', i, q}$ can be estimated via maximum likelihood.
Specifically, for each base $b'$ sequenced with quality $q$ at position $i$ in the read, we can define
the likelihood that this base was actually $b$ by looking at the error probabilities of all $k$-mers that
cover this base.

\begin{equation}
\label{eq:base-likelihood}
\mathcal{L}(b) = \frac{\prod_{j = 1}^{k} p(b | b', i, q) P(\text{kmer}_j^{b, i} \text{ is true})}{\sum_{\hat{b} =
\{A, C, G, T\}} \prod_{j = 1}^{k} p(\hat{b} | b', i, q) P(\text{kmer}_j^{\hat{b}, i} \text{ is true})}
\end{equation}

In equation~\eqref{eq:base-likelihood}, we use $\text{kmer}_j^{b, i}$ as a function that appropriately
substitutes base $b$ into the $j$th $k$-mer that covers the base at position $i$. Also, note in practice
that bases at the start and end of the read will not be covered by $k$ $k$-mers; for example, the first
base of each read is covered by a single $k$-mer. The $P(k\text{-mer is true})$ probabilities
are evaluated using the trie collected by the erroneous $k$-mer detection process described
in~\S\ref{sec:detection}. We note that invalid $k$-mers are not present in that tree; when we search for
a $k$-mer that is not resident in the trie, we substitute the user provided probability cutoff as an upper
bound on the truth probability of that $k$-mer.

To achieve a most accurate estimate for the transition probabilities of error covariate $C_{b', i, q}
\rightarrow p(b | b', i, q)$, we would ideally run an EM algorithm over all bases in $C_{b', i, q}$.
However, this is not  tractable as within a read we cannot separately evaluate the transitions for bases
$i - 1$, $i$, and $i + 1$. To run such an EM algorithm, we would need to evaluate all transitions for all
bases in all reads, which would be computationally intractable; we discuss this in more detail
in~\S\ref{sec:read-refinement}. Instead, we estimate the transition probabilities by treating the
probability estimates of all bases in $C_{b', i, q}$ as the expected outcomes from multinomial trials, and 
apply an MLE estimator. This is equivalent to running a single expectation phase of the canonical EM
algorithm for a mixture of categorical random variables.

\subsection{Read Refinement}
\label{sec:read-refinement}

From~\S\ref{sec:transition-probability}, we have derived the likelihood equations for base
calls~(equation~\eqref{eq:base-likelihood}), and an algorithm for finding the priors for bases in a given
error covariate $C_{b', i, b}$. Given the base likelihood equation and the prior probabilities, we would
like to make corrections to low quality base calls such that we maximize the probability of each read.
We define the probability of each read as:

\begin{equation}
\label{eq:read-probability}
P(\mathbf{r} | \mathbf{r}') = \prod_{i = 1}^l \mathcal{L}(b = r_i)
\end{equation}

Where $\mathcal{L}(b)$ is defined in~\eqref{eq:base-likelihood}. If we consider $\mathbf{r}$ and
$\mathbf{r}'$ to be vectors such that each base of the read exists in it's own coordinate plane, error
correction can be performed via coordinate ascent in the read space. We define the following recursive
algorithm to perform coordinate ascent, where we attempt to correct the read, and only accept this
correction if it improves the probability that the current read sequence is correct:

\begin{algorithm}
\caption{Error Correction via Coordinate Ascent}
\label{alg:ex-coordinate-ascent}
\begin{algorithmic}
\STATE $read \leftarrow$ current read sequence
\STATE $p_{read} \leftarrow$ current read probability
\STATE $\tau \leftarrow$ current priors
\STATE $candidates \leftarrow$ sorted list of bases which are candidates for correction
\IF{$candidates$ is empty}
\RETURN $(read, \tau)$
\ELSE
\STATE $candidate \leftarrow candidates$.first
\STATE $newRead \leftarrow$ correctRead($read, candidate$)
\FORALL{$i \in read$}
\IF{$k$-mers from $i$ overlap $candidate$}
\STATE $\tau'_i \leftarrow p(b | read_i) \tau_i^b, \forall b \in \{A, C, G, T\}$
\ELSE
\STATE $\tau'_i \leftarrow \tau_i$
\ENDIF
\STATE $p'_{read} \leftarrow$ probability($\tau'$)
\IF{$p'_{read} < p_{read}$}
\RETURN recurse($read, p_{read}, \tau, candidates$.dropFirst)
\ELSE
\RETURN recurse($newRead, p'_{read}, \tau',$ computeNewCandidates($newRead, \tau'$))
\ENDIF
\ENDFOR
\ENDIF
\end{algorithmic}
\end{algorithm}

For the first iteration of the algorithm, we use the base transition probabilities
from~\S\ref{sec:transition-probability} as prior probabilities when computing base likelihoods. After the
first iteration, we use the likelihoods from the last accepted step as the priors for the next step. At each
step, we define a base as a candidate for correction if the probability that the base is the ``true'' base at
that site in the read is below a user defined probability~(default is Phred 10 $\rightarrow p(b) < 0.9$),
and if we have not tried this correction \emph{since} the last accepted correction.\footnote{To prevent
oscillating corrections, we also do not allow bases that have been previously edited to be considered as
candidates.} As such, we only need to calculate the correction candidates on the initial iteration, and
after every accepted iteration.

Once the correction algorithm has completed, we perform a final two steps:

\begin{enumerate}
\item We convert the final base probabilities from the coordinate ascent algorithm into Phred-scaled
base quality scores, and
\item We trim low quality bases off of the start and end of the read, where ``low quality'' is defined by a
user provided threshold~(default is Phred 10).
\end{enumerate}

We only attempt to correct a single base per iteration. As mentioned
in~\S\ref{sec:transition-probability}, it is intractable to exhaustively search for correction candidates.
This is because a sequence of length $l$ has $3 {l \choose e}$ correction candidates if we allow up to
$e$ errors to be corrected in a single step. The constant factor of 3 arises because we can correct each 
base to any of the other four bases.

\section{Results}
\label{sec:results}

\subsection{Assembly Accuracy}
\label{sec:assembly}

\subsection{Variant Calling Accuracy}
\label{sec:variant-calling}

\subsection{Performance and Scaling}
\label{sec:performance}

\section{Conclusion}
\label{sec:conclusion}

\subsection{Availability and Reproducibility}
\label{sec:availability}

This read error correction software is made available as part of the \texttt{ADAM} libraries and
command line interface~(CLI)~\cite{massie13}, which are open source under the Apache 2 license.
Our source code is available on GitHub\footnote{\url{https://www.github.com/bigdatagenomics/adam}}.
Additionally, we distribute our libraries through the Apache Maven dependency system's central
repository, with group ID \texttt{org.bdgenomics.adam} and artifact ID \texttt{adam-core}.

Our tests were run on the Amazon Elastic Compute~(EC2) cloud. The scripts used to run our
performance and accuracy tests are publicly accessible via version control on GitHub in the
\texttt{bdg-recipes}\footnote{\url{https://www.github.com/bigdatagenomics/bdg-recipes}} repository, and
are tagged with the \texttt{adam-ec-recomb-15} tag.

\bibliographystyle{splncs03}
\bibliography{kmer-counting}

\end{document}
